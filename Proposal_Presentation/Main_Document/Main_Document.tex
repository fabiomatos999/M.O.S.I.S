\documentclass[14pt, aspectratio=169]{beamer}


\usetheme{CambridgeUS}
\usecolortheme{beaver}
\useinnertheme{rectangles}

\usepackage{bookmark}
\bookmarksetup{
	numbered,
	open,
}
\usepackage{graphicx}

\title{M.O.S.I.S Proposal Presentation}
\author{Fabio J. Matos Nieves \& Eduardo S. Miranda Figueroa }
\institute{Univerity of Puerto Rico Mayagüez Campus}
\date{September 12, 2023}

\begin{document}
\begin{frame}
	\maketitle
\end{frame}

\begin{frame}{Table of Contents}
	\tableofcontents
\end{frame}
\section{Introduction}
\subsection{Problem Background}
\begin{frame}{Project Background}
	\begin{description}
		\item[M.O.S.I.S] Marine Operated Stereoscopic Imaging System
	\end{description}
	\begin{itemize}
		\item Used to study marine life on site
		\item Records environmental conditions
		      \begin{itemize}
			      \item Temperature
			      \item Ph
			      \item Pressure
			      \item Dissolved Oxygen
		      \end{itemize}
		\item Utilizes two cameras: left and right
	\end{itemize}
\end{frame}
\subsection{Market Analysis}
\begin{itemize}
	\item The only device that is similar is the underwater microscope in Andrew D. Mullen et. all.
	\item There is no commercial device for underwater on site microscopy
\end{itemize}
\section{Body}
\subsection{Problem Statement}
\begin{frame}{Problem Statement}
	The microscope already has a user interface, \textbf{but:}.
	\begin{itemize}
		\item lacks design cohesion
		\item lacks support for the on board buttons
		\item crashes frequently
		\item lacks a formal way to store the data captured by the microscope
	\end{itemize}
\end{frame}
\subsection{Deliverables}
\begin{frame}{Deliverables}
	\begin{itemize}
		\item A user interface for the M.O.S.I.S microscope
		\item Adaptation of the existing hardware API
		\item Browsable, automatically generated folder structure for the data and metadata captured by the microscope.
	\end{itemize}
\end{frame}
\begin{frame}{Client and Stakeholders}
	\begin{itemize}
		\item Client
		      \begin{itemize}
			      \item Marine Biology Graduate Student David M. Repollet Otero
		      \end{itemize}
		\item Stakeholders
		      \begin{itemize}
			      \item Marine Biology Graduate Student David M. Repollet Otero
			      \item Dr. Manuel Jiménez.
			      \item Marine Biologist Researchers
			      \item Fabio J. Matos Nieves
			      \item Eduardo S. Miranda Figueroa
		      \end{itemize}
	\end{itemize}
\end{frame}
\begin{frame}{Users}
	\begin{itemize}
		\item Marine Biology Graduate Student David M. Repollet Otero
		\item Marine biologist researchers
	\end{itemize}
\end{frame}
\subsection{Objectives}
\begin{frame}{Objectives}
	\begin{itemize}
		\item By November 27, 2023 we will be delivering:
	\end{itemize}
\end{frame}
\begin{frame}{Objectives: UI}
	\begin{itemize}
		\item Control the microscope
		\item Utilize onboard buttons
		\item Shows a preview of the cameras
		      \begin{itemize}
			      \item 15 FPS
			      \item Downscaled 1920x1080 Resolution
		      \end{itemize}
		\item Display:
		      \begin{table}[h!]
			      \begin{tabular}{|l|l|}
				      \hline
				      Temperature      & 3 significant digits \\ \hline
				      Ph               & 3 significant digits \\ \hline
				      Pressure         & 3 significant digits \\ \hline
				      Dissolved Oxygen & 3 significant digits \\ \hline
			      \end{tabular}
		      \end{table}
	\end{itemize}
\end{frame}
\begin{frame}{Objectives: Hardware API}
	\begin{itemize}
		\item Display camera feed to U.I
		\item Parse sensor data string from UART
		      \begin{itemize}
			      \item Temperature
			      \item Ph
			      \item Pressure
			      \item Dissolved Oxygen
		      \end{itemize}
		\item Run diagnostics at application start.
	\end{itemize}
\end{frame}
\begin{frame}{Objectives: Data Storage}
	\begin{itemize}
		\item Store data and metadata in a browsable format
		\item Format: (Shot Id-Shot Type-Time Stamp).
		      \begin{itemize}
			      \item Left camera media
			      \item Right camera media
			      \item Shot Type
			      \item Time Stamp
			      \item Temperature
			      \item Ph
			      \item Pressure
			      \item Dissolved Oxygen
		      \end{itemize}
	\end{itemize}
\end{frame}
\subsection{Milestones}
\begin{frame}{Milestones}
	\begin{itemize}
		\item The front end by October 15, 2023.
		\item The hardware API by October 30, 2023.
		\item The back end API by November 5, 2023.
		\item The folder structure generator by November 12, 2023.
		\item Connecting the front end with the back end by November 26, 2023.
	\end{itemize}
\end{frame}
\subsection{Budget}
\begin{frame}{Most Significant Budget Items}
	\begin{itemize}
		\item Eduardo S. Miranda \& Fabio J. Matos's Salary - \$20.90 * 30 hours per weeks * 13 weeks  * 2 employees = \$16,302.00
		\item Udemy "Python GUI Development with PyQt6 \& Qt Designer" Class - \$14.99
		\item The Project's Overhead = \$41,703.03
	\end{itemize}
\end{frame}
\section{Conclusion}
\subsection{Outcomes}
\begin{frame}{Outcomes}
	\begin{itemize}
		\item U.I that utilizes the onboard buttons
		      \begin{itemize}
			      \item Start capture with a single button
		      \end{itemize}
		\item Live preview of both cameras
		\item Live temperature, Ph, pressure and dissolved oxygen readings
		\item Separate menu for study selection
		\item Specific button for emergency shutdown.
	\end{itemize}
\end{frame}
\subsection{Project Impact}
\begin{frame}{Project Impact}
	\begin{itemize}
		\item Completes all the software necessary for the microscope to function properly on site.
		\item Streamlines the process of data capture while in the field.
		\item Allows for the data to be in an browsable format.
		\item Facilitates the later analysis of the data.
	\end{itemize}
\end{frame}
\subsection{Questions}
\begin{frame}
	\begin{itemize}
		\item Questions?
	\end{itemize}
\end{frame}
\end{document}