\section*{Executive Summary (Fabio \& Eduardo)}
\begin{spacing}{1.5}

The current U.I for the M.O.S.I.S microscope is unsuitable for on site research, it does not use the on board buttons, crashed frequently and lacks design cohesion. The objectives for our project has not changed, we will deliver an user interface that will allow the users of studying the underwater lifeforms on site. The requirements have not changed. However the client added new requirements, such as formalizing the time stamp format and being able to configure the cameras and sensors at site with the user interface. Due to the microscope not being available from October 15 to October 21  three milestones are being moved around to conform to the delay. Both the back end API and hardware API are being moved up to October 15, while the front end is being moved to October 30. We are still on pace to meet the new October 15 deadlines, for the hardware API and back end API. As for accomplishments we have completed the udemy PyQt6 course, designed the schema for the database and created mock ups for the user interface. After reevaluating the budget, there was an expenditures that were already purchased. This results in a saving of \$11,900.81. The remaining tasks for the project are the back end API by October 15, the hardware API by October 15, the front end by October 30, the folder structure generator by November 7 and integrating front end with back end by November 13.
    
\end{spacing}
\newpage