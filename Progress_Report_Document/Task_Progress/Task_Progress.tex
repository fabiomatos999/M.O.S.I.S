\section{Task Progress (Fabio)}
Upon further analysis, the original schedule was unfeasible since it calculated based on 100\% capacity which is unobtainable in the real world with inevitable delays and happenstance occurring constantly. Upon further consideration, we have adjusted the training load to be 25\% and the implementation load to be 50\%. The reason why the training load is 25\% is that upon starting the course, we have noticed that it takes on average 1.5 times the lesson length to complete a lesson, since we implement what was taught in the video in out own development environment. This helps solidify what was taught in the lesson but has the downside of both increasing training time and developer burn out, thus a 25\% work load was chosen to more accurately reflect how we proceed through the lessons of the PyQt6 course. The 50\% work load during product implementation was chosen to consider real life delays and happenstance but to also not underestimate developer speed and competence. Another side effect of choosing a 50\% work load is we achieve 13 days of lag between the final deadline which allows for a substantial amount of leeway for delays.\\
Up to this point we are progressing through the PyQt6 course at a albeit slow but steady pace. Mainly, we have finished the introduction and about one quarter of the widgets section, which is about 5\% complete of the course according to Udemy statistics.