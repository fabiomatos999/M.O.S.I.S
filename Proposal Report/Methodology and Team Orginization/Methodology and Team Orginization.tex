 \section{Methodology and Team Organization (Fabio)}
 The Scrum methodology was chosen because U.I design necessitates constant feedback from the client, which benefits the scrum development cycle in the following ways:\cite{scrumallianceWhatScrum2015}
 \begin{itemize}
 \item Presents a potentially deliverable build of the U.I to the client after each sprint.
 \item The client can give feedback on the build presented to them.
 \item The feedback can either be added to the sprint backlog or incorporated directly into the next sprint deliverable.
 \item The development cycle focuses on presenting deliverables to the client.
 \item Scrum is a lightweight framework of the Agile methodology which allows for self-organization and adaptability.
 \end{itemize}
 To accomplish the project using Scrum we first create a sprint backlog from the objectives signed by the client.\cite{scrumallianceWhatScrum2015} Then a sprint is planned choosing from the sprint backlog, where the Scrum master helps decide what deliverables can be accomplished within a given sprint.\cite{scrumallianceWhatScrum2015}\cite{eyeontechWhatScrumMaster2020} While in the sprint, the Scrum master has daily meeting with the team and asks 3 questions:\cite{eyeontechWhatScrumMaster2020}
 \begin{enumerate}
 \item What did you do yesterday?
 \item What will you today?
 \item What problems impede your progress?
 \end{enumerate}
 The Scrum master then helps the team reach consensus, keeps the team focused and removes any obstacles impeding progress.\cite{eyeontechWhatScrumMaster2020}\\
 For testing and quality assurance for the U.I, we will be meeting with the client on a weekly basis to assure that the U.I meets their standards and have the client test the U.I in meeting. To supplement this, during development, pair programming will be employed in order to reduce bugs and to test software by a different person that is not the developer. For the back-end elements, unit and integration tests will be extensively used to ensure that the expected output is correct, for both single modules and the interface between modules. This strategy reduces the time for successful module integration.\\
 The work will be broken down as follows:
 \begin{enumerate}
 \item The client creates a sprint backlog.
 \item A sprint is planned with the Scrum master helping in deciding if the tasks within the sprint are feasible.
 \item At the beginning of the day during the sprint, a stand up meeting is held with the team and the Scrum master focuses the team along with mediating and resolving problems that impede progress.
 \item A potentially deliverable build of the U.I is shown to the client during the weekly meeting.
 \item The client gives feedback on the build shown to them.
 \item The feedback is incorporated into either a new sprint item or address into the next sprint.
 \item Repeat steps 2-6 until all items from the sprint backlog have been completed.
 \end{enumerate}
 For team organization, Fabio J. Matos Nieves will be both Scrum master, project manager and lead hardware API developer while Eduardo S. Miranda Figueroa will be the lead back-end developer.