\section{Outcomes (Eduardo)}
M.O.S.I.S UI 2.0 will allow the researcher to execute the functions of the microscope with the onboard buttons and can start capturing data with a single button. The onboard screen will present the data captured by the sensors such as temperature, Ph, dissolved oxygen and pressure, and show a live feed of the cameras. In a separate menu the researcher will be able to preview and select a pre-configured study profile of their choosing. After a study is selected the researcher will be able to start/stop the study and shut off the microscope safely. We will be building the new UI with weekly feedback received from the client, adjusting the UI as needed.\\
The new UI will need to connect with the microscope's hardware to gather the necessary data. The adaptation of the existing hardware API will allow the UI to show the live feed footage from the cameras, show updated sensor data and run diagnostics sub-routines at start-up to troubleshoot the hardware. Diagnostics will be executed when the UI starts.\\
The data and metadata gathered by the sensors will then be stored in a standardized format suitable for browsing. The researcher will be able to see the data captured in a file browser. As a added bonus the data captured by the microscope can be analyzed by an external tool. 