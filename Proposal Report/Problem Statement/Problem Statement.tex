 \section{Problem Statement (Fabio)}
 The Marine Operated Stereoscopic Imaging System (M.O.S.I.S) is an underwater microscope research project developed by Marine Biology Graduate Student David M. Repollet Otero and Dr. Manuel Jiménez.\cite{fernandoguzman3UIMosis2023}\cite{Fabiomatos999} The M.O.S.I.S microscope is used to analyze aquatic habitats \textit{in situ}, since many of the conditions necessary to analyze these environments cannot be replicated in a laboratory setting. The microscope has all of the hardware functions installed and the sensors collect data correctly by one of the two on board microcontrollers.\\ The current M.O.S.I.S microscope has a user interface (U.I) that was developed by a group of various volunteers throughout various semesters. The main issue with the current U.I is that it lacks design cohesion, lacks support for the use of the on board buttons, crashes frequently, and lacks a formal way to store the data captured by the microscope. This is a problem because the microscope will be used underwater by researchers, thus making the current version unusable for that purpose.\\
 The customer for this project is Marine Biologist Graduate Student David M. Repollet Otero. The users will be Graduate student David himself along with other marine biologist researchers working in the field. The stakeholders for this project are Graduate Student David M. Repollet Otero, Dr. Manuel Jiménez, the marine biologist researchers that will be using this U.I along with Fabio J. Matos Nieves and Eduardo S. Miranda Figueroa.\\
 M.O.S.I.S U.I 2.0 proposes to completely re-write the current U.I from scratch in order to ensure a cohesive and tested U.I that can be trusted for \textit{in situ} research. This not only ensures that the research with M.O.S.I.S microscope can be done smoothly but also allows for the media captured by the microscope to be in a standardized format that can even be later analyzed and presented to the researcher automatically.\\
The scope of the project is to re-write the existing M.O.S.I.S microscope U.I from scratch and to redesign the file management systems currently employed by the microscope to be in a standard format.