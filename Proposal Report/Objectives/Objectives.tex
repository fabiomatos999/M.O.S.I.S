\section{Objectives (Eduardo)}
%%Que vas hacer, como lo vas a medir y para cuando SER ESPECIFICO
\begin{enumerate}
	\item Design a UI for the M.O.S.I.S project Raspberry Pi by November 27, 2023 to:
	      \begin{itemize}
		      \item Control the functions of the microscope, utilizes the onboard buttons, shows a live preview at 15 fps at a downscale resolution of 1920 X 1080
		      \item Display temperature sensors data with an accuracy of 3 significant digits
		      \item Display Ph sensor data with an accuracy of 3 significant digits
		      \item Display pressure sensor data with an accuracy of 3 significant digits.
		      \item Display dissolved oxygen sensor data with an accuracy of 3 significant digits
		      \item Display current local IP address
		      \item Choose the configuration file for the specific study to be done
		      \item Achieving a tenfold decrease in transition time in comparison to the existing user interface when switching through different windows in the user interface
	      \end{itemize}
	\item Adapt the currently existing hardware API, by November 27, 2023 to:
	      \begin{itemize}
		      \item Display the live feed from the cameras to the U.I
		      \item Parse the temperature, Ph, pressure and dissolved oxygen data string from the UART port
		      \item Run diagnostic sub-routines at application start
	      \end{itemize}
	\item By November 27, 2023, store data and metadata in a format that a researcher can browse through in a file browser, of which include:
	      \begin{itemize}
		      \item Left camera media
		      \item Right camera media
		      \item Shot type
		      \item Time stamp
		      \item Temperature
		      \item Ph
		      \item Pressure
		      \item Dissolved oxygen
	      \end{itemize}
\end{enumerate}