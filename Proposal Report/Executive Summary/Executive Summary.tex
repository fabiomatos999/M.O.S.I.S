 \section*{Executive Summary}
 The existing user interface on the M.O.S.I.S microscope is currently unsuited for current state of the project, is unwieldy and unreliable for use in the field. This is further compounded by the fact that since the M.O.S.I.S project is aimed at marine researchers and biologists, having such an unpolished user interface is unacceptable. Marine Biology Graduate Student David M. Repollet Otero and Dr. Manuel Jiménez are the major stakeholders in this remake and they are expecting a user interface for the M.O.S.I.S microscope that is easy to use in the field while still providing the basic functionality of the microscope such as capturing media, switching the type of study being performed and the safe shutdown of the system. By the end of the semester it is expected that a user interface, hardware API and a standardized folder structure for the captured media be delivered that will comprise a whole product known as M.O.S.I.S U.I  2.0. The key milestones for this project are the front end of the user interface, the adaptation of the current hardware API, the automated construction of the folder structure from the database and lastly the integration between the front end and back end. The total cost of the project is \$27,802.02 where \$6,172.01 corresponds to the M.O.S.I.S microscope. M.O.S.I.S U.I 2.0 will complete the required software for the microscope, while streamlining the process of data capture on the device.