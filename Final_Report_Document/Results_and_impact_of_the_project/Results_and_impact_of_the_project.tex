\section{Results and impact of the project (Eduardo)}
M.O.S.I.S completed the project in the expected time frame. The technical results of the project are: allowing the microscope to be controlled with the on board hall effect sensors. The user interface also permits the user to control the LEDs on the microscope and configure the cameras as per the studies. As for the ethical aspects of the project, the values of the project were efficiency, open source software and the environment. The value of efficiency at the core of out project by the means of automating capture of data while the microscope is doing field work.  The project is open source and does not require any confidentiality. The open source philosophy also promotes cooperation and openness in programming.  With the environmental aspects of the project the risks are affecting the corals and other marine species both positively in the sense that environmental research can be conducted on marine life to analyze the impacts of climate change, however in the negative sense the instrument itself could cause damage to the immediate environment where it is being used by utilizing lighting that is not natural to the habitat. We have taken precautions with the lights only turning them on during studies and for time lapse shot type, activating the light only when needed. For the social aspects of the project it will allow the researchers to get better data on the bleaching on the corals and will let them publish the results with other researchers and subsequently inform the public on the environmental phenomena. The legal aspects of the project are governed by the GPLv3 license which is an open source license that allows for modification, distribution, patent use, private and commercial use, while waving any liability and warranty rights.