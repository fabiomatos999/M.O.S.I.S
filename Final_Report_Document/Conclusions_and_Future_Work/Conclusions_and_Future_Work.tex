\section{Conclusions and Future Work (Fabio)}
\subsection{Conclusions}
In conclusion, M.O.S.I.S UI 2.0 was able to complete the integration of the hardware components of the microscope into a functional prototype that can be used to further environmental research. The complementing nature of the team members was a great advantage during implementation and integration. Each member worked on the aspects of the project that that they were most comfortable, be it hardware, software or databases, while not straying away from learning new concepts such as native desktop application development. The project was successfully demonstrated to the client, stakeholder and the capstone professor on schedule meeting all of the amended requirements. While there were major delays in the integration and hardware API, the project time table was flexible enough to allows for the unforeseen delays. With the M.O.S.I.S microscope operational, it can be used in the field to directly contribute to environmental research, hastening the collection of valuable research data and expediting scientific publications on the impacts of climate change which can inform the public on the current state of the planet.

\subsection{Future Work}
There are various improvements that can be done to improve the M.O.S.I.S UI 2.0, those being:
\begin{enumerate}
	\item Increase the frame rate of the preview by using the Pixelink SDK preview functions instead of doing manual processing on captured images.
	\item Display button controls for each menu
	\item Have the onboard buttons have hold functionality.
	\item Implement PWM for on-board lighting.
	\item Implement focus step manipulation using the on-board buttons.
	\item Implement auto focus for the cameras.
	\item Focus point visualizer.
	\item Visible preview when adjusting camera settings.
	\item Standardize gain to use ISO measurements.
	\item Swap left and right camera to reflect stereoscopic capture.
	\item Lighting toggle on preview screen.
\end{enumerate}
\subsection{Lessons Learned}
Given that the M.O.S.I.S microscope has already been in development for 4 years, either by the client and the volunteers that have contributed to the hardware and software respectively, the importance of proper and detailed documentation has never been more apparent. If the camera SDK had an explicit message saying that the camera feeds cannot be embedded within the application inside of Linux, then the entire design of the user interface would have changed. Just a single line of documentation would have increased the usability of the microscope drastically and the frame rate issues that have plagued this iteration of the user interface would have been mostly eliminated. On a related note, we learned the importance of having updated schematics for existing hardware. Due to the lack of up to date schematics for the vast majority of the existing hardware, integration slowed down significantly. In a more positive note, holding the client to the signed requirements was of great help during integration since the client always wants more functionality, but the requirements are made for the project timetable given to us, not more nor less. This helped us greatly in maintaining focus on the project objectives instead of getting bogged down more ever mounting feature requests and this resulted in our project fulfilling all of the amended project requirements. Another realization during this project was just how important knowing embedded systems is for a computer engineer. If it neither of us had no experience with embedded systems, this project would be impossible given the scope of the requirements and almost non existent documentation. However, the most important lesson learned throughout this semester was the important of working in a team. The amount of work in this project is simple too much for a single person to do in less than 5 months time, thus maintaining good relationships with your teammates while also keeping them doing their work is of utmost importance since the only thing that matters at the end of capstone is the final prototype. This taught us how to cooperate with not just our team but other teams, fostering a collaborative development environment, which is crucial for any effective engineer.