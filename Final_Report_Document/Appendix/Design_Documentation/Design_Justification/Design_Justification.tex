\subsection{Design Justification (Eduardo)}
\subsubsection{Use Case Diagram}
Since this is a user facing application, the bulk of the use cases are attributed to the users. However, the reason why the application itself is responsible for performing diagnostics is to avoid an invalid state for the user interface to start. The use cases were derived from the requirements from the client and stakeholder.
\subsubsection{Class Diagram}
The class diagram for the system is broken up into two parts: The UI classes and the utility classes. The UI classes demonstrate how the main menu and the sub menus relate to each other, in this case a dependency relationship where the sub menus cannot exist without the main menu and there can be only one of each one of the sub menus at any given time. The utility classes are classes that are instantiated by the sub menu classes but do not exhibit any dependency, composition or inheritance relationships.
\subsubsection{U.I Mockups}
The color scheme that we chose for the background is black with neon orange text and gray background with neon green text. The reason why the background colors for the U.I mockups are gray and black are because we utilized studies that show that the color black and gray looks darker underwater.\cite{AquaticSafetyGroup2021}\cite{FluoGreenMost2018} Likewise the same studies show that bright colors such as neon orange and neon green looks equally as bright underwater.\cite{AquaticSafetyGroup2021}\cite{FluoGreenMost2018} This insures that the researchers and divers have a high contrast user interface in environments with poor visibility.